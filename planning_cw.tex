\documentclass[a4paper]{article}

\usepackage{parskip}
\usepackage{setspace}
\usepackage{fullpage}
\usepackage{graphicx}
\usepackage{float}

\begin{document}

\title{Planning, Estimating and Tracking}
\author{Andrew Higginson \and Bryan Liu \and Jia Guang Choo \and Emma Hulme \and 
Timothy van Bremen \and Thomas Taylor-Hall}
\date{\today}
\maketitle

\setcounter{table}{0}
\linespread{1.1}

\section{Project Introduction}
As part of the renovation of the William Penney Building on the Sherfield 
walkway, interactive screens are to be installed. Mounted inside the building, four projectors will simultaneously display content onto floor to ceiling glass
panels that are visible to passers-by. Also, an 84-inch 4K resolution touch 
screen is to be mounted by the entrance doors. 

Our project consists of developing an ``App Store" for uploading interactive content and visualisations to be displayed on the four projected screens. Also, we will be developing a playout system to show the content on multiple screens in multiple resolutions.

%Current plan and how we can adapt it:
%	- agreed on estimation of amount of time it takes to ocmplete tasls
%  - timeline
%  - leeway for problems (experience from previous projects)

%How did we come up with it? 
%	- identifying components
%	- taking into account other 

\section{Understanding User Requirements}

\begin{figure}[H]
  \centering
    \includegraphics[width = 0.55\textwidth, trim= 0 0.55cm 0 1.4cm, clip]{./planning/userreq.jpg}

  \caption{Initial project specification in a diagram.}
  \label{fig:userreq}
\end{figure}

To better understand user requirements and form the initial backlog, we have
held an initial meeting with our supervisor, who in this instance plays the role of ``client'' of the 
project.

The ``App Store" is broadly divided into the following three domains:
\begin{itemize}
  \item \textbf{Submission}: allowing users to login, view and submit 
        visualisations/ advertisements
  \item \textbf{Moderation \& Scheduling}: allowing administrators to approve
        and schedule submissions for playout
  \item \textbf{Playout}: allowing scheduled submissions to be projected in
        specified sizes

\end{itemize}



Figure \ref{fig:userreq} illustrates requirements of the system.
The full specification is also available in the appendix.

\section{Task Estimation}
The group is acutely aware that with Computing examinations being held at the
end of term, we could only practically carry out development work until the
first week of December. Furthermore, commitments in other courses mean that 
each team member is only able to devote around 16-20 hours per week to
the project.

In light of these constraints, we have decided to commit to the following:
\begin{enumerate}
  \item \textbf{To maintain a one-week iteration}: this allows the development team to
        obtain maximum possible feedback from the client/supervisor.
  \item \textbf{To strictly adhere to the original project scope}: while we believe the
        current scope is manageable, we would reject time-consuming items which
        are out of scope before a minimal working system is implemented.
\end{enumerate}


\begin{figure}[h]
  \centering
    \includegraphics[width = 0.99\textwidth]{./planning/timeline.jpg}
   
  \caption{Project plan with timeline}
  \label{fig:timeline}
\end{figure}


\begin{table}[h]
  \begin{tabular}{c | c | c | c }
    \textbf{Academic Week} & 3 \& 4 (-30/10) & 5 (-6/11) \\
    \textbf{Iteration/Sprint} & 1 & 2 \\ \hline
    \textbf{Tasks} & Front-page + grid view (FE) & Access request for externals \\
          & Docker Setup (BE)           & Visualisation submission \\
          & Hello-world server (Flask) (BE) & \\
          & MongoDB Setup + Integration (BE) & \\
          & Jenkins Setup at Production VM (BE) & \\
          & Internal Kerberos Login (BE) & \\
  \end{tabular}

  \vspace{30pt}
  \begin{tabular}{c | c | c | c | c}
    \textbf{Academic Week} & 6 (-13/11) & 7 (-20/11) & 8 (-27/11) & 9 (-4/12) \\
    \textbf{Iteration/Sprint} & 3 & 4 & 5 & 6 \\ \hline
    \textbf{Tasks} & Comment/view existing visualisations & Scheduling & 
            \multicolumn{2}{c}{Playout} \\
          & Admin moderation/ approval &  \\
  \end{tabular}
  \caption{Project timeline}
  \label{table:timeline}
\end{table}

% Estimation & Planning - Iteration Plans + Release Plans
The resultant plan and timeline shown in figure \ref{fig:timeline} and table
\ref{table:timeline} is a
combination of the plan for next iteration and the release plan, taking account
of the current project scope. Each post-it note either represents a 
concrete task to be completed during the next iteration (towards the left), or
high level themes to be implemented (towards the right). Each line segment at
the bottom represents an iteration ending on Thursday, when the development
team will meet with the client.

% Estimation & Planning - T-Shirt Sizing
Estimates on time required for the tasks were based on their relative size and
time taken for us to complete similar ones in the past. For example, some of us
have implemented a College (Kerberos) Login System well within a week, thus if
it is a size M, we can infer that the team (now double the size) is capable in 
fitting two size M tasks within an iteration. For larger system modules, we
assign a longer period specifically for that task: we expect the scheduling
system (size L) and the entire playout system (size XL) would take us one and
two iteration(s) respectively.


\section{Group organisation} \label{sec:group}
After we established the requirements of the project, each group member stated
which part of the project they would like to work on. We found that there was a
good split of two people that wanted to work on the frontend, two on the backend
server code, and two on the database.

Although this is a good split to initiate work, we realised that the frontend 
aspect of the project may require more work approaching later iterations.
In addition, we  expect that the server code and database should be fully 
implemented within the first four iterations, only requiring minor fixes thereafter.

Therefore, we decided that two people from the backend would move on to creating
the playout software on the dedicated computers; one person would help with the
frontend and the remaining person would apply small fixes and refactors to the
existing server/database code. 


\section{Development Methods \& Project Tracking}
The team has agreed to adopt a mixture of Agile Development Methodologies to
sure our practical need.

For project management, we generally follow the scrum method, adopting the
following characteristics:
\begin{itemize}
  \item \textbf{Roles}: We see Dr. David Birch, our supervisor, as the \textit{Product Owner}
        who provide ideas and feedback for the development. At the same time,
        the team have agreed that Andrew and Bryan should assume the role of
        \textit{Scrum Masters} to facilitate work and all team members should organise
        work for themselves.
  \item \textbf{Ceremonies}: Upon the initial meeting, we have established a
        meeting with David each Thursday to review our previous sprint 
        and plan our next sprint. We are hold a development team
        meeting every Monday (figure \ref{fid:scrum} to update each other 
        on our progress and review work done in past sprints.
  \item \textbf{Artefacts}: We keep our product backlog and iteration backlog on
        Trello as part of our project tracking mechanism (more details
        available in the later part of section).
\end{itemize}


\begin{figure}[ht]
  \centering
    \includegraphics[width = 0.99\textwidth]{./planning/scrum.jpg}
   
  \caption{Half-weekly "scrum" (development team meeting).}
  \label{fig:scrum}
\end{figure}

We also integrate a few technical practices inspired by Extreme Programming:
\begin{itemize}
  \item \textbf{Simple Design}: including the use of lightweight technologies that are easy to understand (this was key to our choice of Flask as a backend framework)
  \item \textbf{Pair Programming}: as mentioned in section \ref{sec:group}, we sorted
        ourselves into three pairs, focusing on frontend (FE), backend (BE) and database (DB) work.
        Such a practice allows continuous development on all divisions without
        being affected by instances in which a member is required to temporarily shift his or her
        focus from the project to other coursework/tests.
  \item \textbf{Test-driven Development}: %TODO

\end{itemize} 

% - Figure on TDD

Finally, to keep track on the group's progress, we use an electronic task
board on Trello. %(+doc, checklist, photo easily add-able)
%  - using trello (physical story board not feasible, kanban)
%  - constantly communicating (messaging software, regular stand-ups)

% Why not follow other practice of kanban/XP/Scrum


\begin{figure}[ht]
  \centering
    \includegraphics[width = 0.99\textwidth]{./planning/trello.jpg}
   
  \caption{Using Trello to keep track of progress.}
  \label{fig:trello}
\end{figure}

\section{Development Tools}
% 	- Flask for the backend as it's new and interesting, lightweight
%	- bryan wanted to experiment with MongoDB%
%	- AngularJS for frontend for interest's sake
%	- Docker as we want to deploy on multiple virtual machines using cloudstack
%	- can test backend if frontend has changed and is not working
 % - David's VM
%	- dedicated comps to run playout software



\end{document}
