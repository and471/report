\documentclass{article}
\begin{document}

\title{Planning, Estimating and Tracking}
\author{Andrew Higginson \and Bryan Liu \and Choo Jia Guang \and Emma Hulme \and Timothy 'The Power' van Bremen \and Thomas Taylor-Hall}
\maketitle

\section{Project Introduction}
As part of the renovation of the William Penery Building on the Sherfield 
walkway, interactive screens are to be installed. Mounted inside the building, 4
projectors will simulataneously display content onto floor to ceiling glass
panels that are visible to passers-by. Also, and 84-inch 4K resolution touch 
screen is to be mounted by the entrance doors. 

Our project consists of developing an "App Store" for uploading images and 
interactive content to be displayed on the 4 projected screens. Also, we will be
developing a playout system to show the content on multiple screens in multiple 
resolutions.

\section{Specification}
SEE GOOGLE DOC FOR THIS!

\section{Tools Used}
 	- Flask for the backend as it's new and interesting, lightweight
	- bryan wanted to experiment with MongoDB
	- AngularJS for frontend for interest's sake
	- Docker as we want to deploy on multiple virtual machines using cloudstack
	- can test backend if frontend has changed and is not working
  - David's VM
	- dedicated comps to run playout software

\section{Project Plan}
Current plan and how we can adapt it:
	- agreed on estimation of amount of time it takes to ocmplete tasls
  - timeline
  - leeway for problems (experience from previous projects)
  - using trello
  - constantly communicating
How did we come up with it? 
	- identifying components
	- taking into account other 

After we established the requirements of the project, each group member stated
which part of the project they would like to work on. We found that there was a
good split of two people that wanted to work on the frontend, two on the backend
server code, and two on the database.

Although this is a good split to initiate work, we realised that the frontend 
aspect of the project may require more work in the long term. In addition, we 
expect that the server code and database should be fully implemented a few
weeks before the frontend is completed, only requiring minor fixes.

Therefore, we decided that two people would move on to creating the playout 
software on the dedicated computers; one person would help with the frontend and
the remaining person would apply small fixes and refactors to the existing 
server/database code. 


\section{Team Management}


\section{Software Engineering Methodology}

\end{document}
