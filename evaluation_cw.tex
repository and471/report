\documentclass[a4paper]{article}

\usepackage{parskip}
\usepackage{setspace}
\usepackage{fullpage}
\usepackage{graphicx}
\usepackage{float}

\begin{document}

\title{Product Management, Feedback and Evaluation}
\author{Andrew Higginson \and Bryan Liu \and Jia Guang Choo \and Emma Hulme \and 
Timothy van Bremen \and Thomas Taylor-Hall}
\date{\today}
\maketitle

\setcounter{table}{0}
\linespread{1.15}

\section{Project Introduction}
As part of the renovation of the William Penney Building on the Sherfield 
walkway, interactive screens are to be installed. Mounted inside the building, four projectors will simultaneously display content onto floor to ceiling glass
panels that are visible to passers-by. Also, an 84-inch 4K resolution touch 
screen is to be mounted by the entrance doors. 

Our project consists of developing an ``App Store" for uploading interactive content and visualisations to be displayed on the four projected screens. Administrators will also be able to use this system to moderate and schedule content. Finally, we will be developing a playout system to show the content on multiple screens in multiple resolutions.


\section{Requirements and Managing Tasks}
In the first week of beginning our project, we met with our supervisor 
David to draw out the initial components. Initially, these were general
requirements that we divided into user, admin and interface 
requirements. We later sent these requirements to David so he could check
they were satisfactory and have them for his records. He was happy with 
these and did not modify them.

TODO: screen of requirements

In the coming weeks, we then clarified and expanded these requirements as 
a group. This allowed us to discuss implementation and technical details
of specific features.

During development, we are constantly looking at the requirements, which 
we have stored in a shared document. In our weekly scrums, we have 
presented the work we have done and discuss whether the project is on the 
right track. In our discussions, David has refined how he wants our
scheduling algorithm to schedule visualisations for a particular time. 

We have been prioritising tasks using our Trello board with different
columns. In addition to this, we have been constantly communication both 
in person and online, to make sure members are implementing assigned tasks
in appropriate times. We have been actively encouraging use of the Trello 
board to update the group when tasks have been completed. This way, 
we don't have to constantly ask or look at code to see if a feature has 
been implemented. If appropriate, group members working on the backend
have been using an internal wiki on gitlab to provide information about
routing, controller actions and parameters. 

TODO: Trello priority columns screen


\section{Evaluation of Project}
%How will we evaluate?
As a group, we are constantly evaluating our project by testing after a 
feature is implemented. We make use of both RSpec and manual testing. In 
addition, our project progress is evaluated by David in our weekly 
meetings. Using Jenkins for Continuous Integration, we have given David
the address of our ``release vm'', where he can see the latest working 
version of the project. 

Quantitively, we can evaluate our project by ticking off our intial 
requirements. Also, we will use the systems ourselves, both from the user 
and admin perspective, to evaluate the project.

TODO: get externals to evaluate
TODO: stuff from eval lecture



\section{Relationships and Feedback}
Our relationship with our supervisor David has consisted of weekly 
meetings and emails if appropriate. Before designing a particular screen
or user interface feature, we produce mockups. We then give these to 
David, who gives his opinion. After the next sprint cycle, we show the 
implemented screens. 

We have also got some representative model data to seed our databse, 
instead of using dummy data. This was provided by David and will also be 
used to test submission, moderation and display features of our system.


TODO: Give specific example of what he's said
TODO: Picture


\end{document}
